\documentclass[a4paper]{article}

% Language setting
\usepackage[spanish]{babel}

%\usepackage{biblatex} si quisiera añadir un sample.bib

\usepackage[top=2cm,left=3cm,right=3cm,marginparwidth=1.75cm]{geometry}


%useful packages
\usepackage{amsmath}
\usepackage{graphicx}
\usepackage[colorlinks=true, allcolors=blue]{hyperref}

\title{Documento test}
\author{Miguel V}
\date{\today}

\begin{document}
\maketitle

\begin{abstract}
    Resumen
\end{abstract}

\section{Intro}

Simplemente se escribe la info textual para las secciones (lo mismo si tuviera abstract antes).

\section{Seccion de prueba 2}
\subsection{Subseccion 1 (inclusion de figuras)}

\begin{figure}
    \centering
    \includegraphics[width=0.3\textwidth]{pfp_image.png}
    \caption{\label{fig:pfp_image}Imagen pfp de test subida desde el mismo directorio.}
\end{figure}


\subsection{How to Make Tables}

Usar los comandos de tabla y tabular para las tablas basicas. Ver por ejemplo: Table~\ref{tab:widgets}.

\begin{table}
    \centering
    \begin{tabular}{l|r}
        Item & Quantity \\\hline
        Widgets & 42 \\
        Gadgets & 13
    \end{tabular}
    \caption{\label{tab:widgets}An example table.}
\end{table}

\subsection{How to Write Mathematics}

\LaTeX{} is great at typesetting mathematics. Let $X_1, X_2, \ldots, X_n$ be a sequence of independent and identically distributed random variables with $\text{E}[X_i] = \mu$ and $\text{Var}[X_i] = \sigma^2 < \infty$, and let
$$S_n = \frac{X_1 + X_2 + \cdots + X_n}{n}
      = \frac{1}{n}\sum_{i}^{n} X_i$$
denote their mean. Then as $n$ approaches infinity, the random variables $\sqrt{n}(S_n - \mu)$ converge in distribution to a normal $\mathcal{N}(0, \sigma^2)$.


\subsection{How to Make Lists}

You can make lists with automatic numbering \dots

\begin{enumerate}
    \item Like this,
    \item and like this.
\end{enumerate}
\dots or bullet points \dots
\begin{itemize}
    \item Like this,
    \item and like this.
\end{itemize}
\dots or with words and descriptions \dots
\begin{description}
    \item[Word] Definition
    \item[Concept] Explanation
    \item[Idea] Text
\end{description}



\end{document}